%%%%%%%%%%%%%%%%%%%%%%%%%%%%%%%%%%%%%%%%%%%%%%%%%%%%%%%%%%%%%%%%%%%%%%%%
%%%%%%%%%%%%%%%%%%%%%% Simple LaTeX CV Template %%%%%%%%%%%%%%%%%%%%%%%%
%%%%%%%%%%%%%%%%%%%%%%%%%%%%%%%%%%%%%%%%%%%%%%%%%%%%%%%%%%%%%%%%%%%%%%%%

%%%%%%%%%%%%%%%%%%%%%%%%%%%%%%%%%%%%%%%%%%%%%%%%%%%%%%%%%%%%%%%%%%%%%%%%
%% NOTE: If you find that it says                                     %%
%%                                                                    %%
%%                           1 of ??                                  %%
%%                                                                    %%
%% at the bottom of your first page, this means that the AUX file     %%
%% was not available when you ran LaTeX on this source. Simply RERUN  %% 
%% LaTeX to get the ``??'' replaced with the number of the last page  %% 
%% of the document. The AUX file will be generated on the first run   %%
%% of LaTeX and used on the second run to fill in all of the          %%
%% references.                                                        %%
%%%%%%%%%%%%%%%%%%%%%%%%%%%%%%%%%%%%%%%%%%%%%%%%%%%%%%%%%%%%%%%%%%%%%%%%

%%%%%%%%%%%%%%%%%%%%%%%%%%%% Document Setup %%%%%%%%%%%%%%%%%%%%%%%%%%%%

% Don't like 10pt? Try 11pt or 12pt
\documentclass[10pt]{article}
\usepackage[utf8]{inputenc}

% This is a helpful package that puts math inside length specifications
\usepackage{calc}

% Layout: Puts the section titles on left side of page
\reversemarginpar

%
%         PAPER SIZE, PAGE NUMBER, AND DOCUMENT LAYOUT NOTES:
%
% The next \usepackage line changes the layout for CV style section
% headings as marginal notes. It also sets up the paper size as either
% letter or A4. By default, letter was used. If A4 paper is desired,
% comment out the letterpaper lines and uncomment the a4paper lines.
%
% As you can see, the margin widths and section title widths can be
% easily adjusted.
%
% ALSO: Notice that the includefoot option can be commented OUT in order
% to put the PAGE NUMBER *IN* the bottom margin. This will make the
% effective text area larger.
%
% IF YOU WISH TO REMOVE THE ``of LASTPAGE'' next to each page number,
% see the note about the +LP and -LP lines below. Comment out the +LP
% and uncomment the -LP.
%
% IF YOU WISH TO REMOVE PAGE NUMBERS, be sure that the includefoot line
% is uncommented and ALSO uncomment the \pagestyle{empty} a few lines
% below.
%

%% Use these lines for letter-sized paper
\usepackage[paper=letterpaper,
            %includefoot, % Uncomment to put page number above margin
            marginparwidth=1.2in,     % Length of section titles
            marginparsep=.05in,       % Space between titles and text
            margin=1in,               % 1 inch margins
            includemp]{geometry}

%% Use these lines for A4-sized paper
%\usepackage[paper=a4paper,
%            %includefoot, % Uncomment to put page number above margin
%            marginparwidth=30.5mm,    % Length of section titles
%            marginparsep=1.5mm,       % Space between titles and text
%            margin=25mm,              % 25mm margins
%            includemp]{geometry}

%% More layout: Get rid of indenting throughout entire document
\setlength{\parindent}{0in}

%% This gives us fun enumeration environments. compactitem will be nice.
\usepackage{paralist}

%% Reference the last page in the page number
%
% NOTE: comment the +LP line and uncomment the -LP line to have page
%       numbers without the ``of ##'' last page reference)
%
% NOTE: uncomment the \pagestyle{empty} line to get rid of all page
%       numbers (make sure includefoot is commented out above)
%
\usepackage{fancyhdr,lastpage}
\pagestyle{fancy}
%\pagestyle{empty}      % Uncomment this to get rid of page numbers
\fancyhf{}\renewcommand{\headrulewidth}{0pt}
\fancyfootoffset{\marginparsep+\marginparwidth}
\newlength{\footpageshift}
\setlength{\footpageshift}
          {0.5\textwidth+0.5\marginparsep+0.5\marginparwidth-2in}
\lfoot{\hspace{\footpageshift}%
       \parbox{4in}{\, \hfill %
                    \arabic{page} of \protect\pageref*{LastPage} % +LP
%                    \arabic{page}                               % -LP
                    \hfill \,}}

% Finally, give us PDF bookmarks
\usepackage{color,hyperref}
\definecolor{darkblue}{rgb}{0.0,0.0,0.3}
\hypersetup{colorlinks,breaklinks,
            linkcolor=darkblue,urlcolor=darkblue,
            anchorcolor=darkblue,citecolor=darkblue}

%%%%%%%%%%%%%%%%%%%%%%%% End Document Setup %%%%%%%%%%%%%%%%%%%%%%%%%%%%


%%%%%%%%%%%%%%%%%%%%%%%%%%% Helper Commands %%%%%%%%%%%%%%%%%%%%%%%%%%%%

% The title (name) with a horizontal rule under it
%
% Usage: \makeheading{name}
%
% Place at top of document. It should be the first thing.
\newcommand{\makeheading}[1]%
        {\hspace*{-\marginparsep minus \marginparwidth}%
         \begin{minipage}[t]{\textwidth+\marginparwidth+\marginparsep}%
                {\large \bfseries #1}\\[-0.15\baselineskip]%
                 \rule{\columnwidth}{1pt}%
         \end{minipage}}

% The section headings
%
% Usage: \section{section name}
%
% Follow this section IMMEDIATELY with the first line of the section
% text. Do not put whitespace in between. That is, do this:
%
%       \section{My Information}
%       Here is my information.
%
% and NOT this:
%
%       \section{My Information}
%
%       Here is my information.
%
% Otherwise the top of the section header will not line up with the top
% of the section. Of course, using a single comment character (%) on
% empty lines allows for the function of the first example with the
% readability of the second example.
\renewcommand{\section}[2]%
        {\pagebreak[2]\vspace{1.3\baselineskip}%
         \phantomsection\addcontentsline{toc}{section}{#1}%
         \hspace{0in}%
         \marginpar{
         \raggedright \scshape #1}#2}

% An itemize-style list with lots of space between items
\newenvironment{outerlist}[1][\enskip\textbullet]%
        {\begin{itemize}[#1]}{\end{itemize}%
         \vspace{-.6\baselineskip}}

% An environment IDENTICAL to outerlist that has better pre-list spacing
% when used as the first thing in a \section 
\newenvironment{lonelist}[1][\enskip\textbullet]%
        {\vspace{-\baselineskip}\begin{list}{#1}{%
        \setlength{\partopsep}{0pt}%
        \setlength{\topsep}{0pt}}}
        {\end{list}\vspace{-.6\baselineskip}}

% An itemize-style list with little space between items
\newenvironment{innerlist}[1][\enskip\textbullet]%
        {\begin{compactitem}[#1]}{\end{compactitem}}

% To add some paragraph space between lines.
% This also tells LaTeX to preferably break a page on one of these gaps
% if there is a needed pagebreak nearby.
\newcommand{\blankline}{\quad\pagebreak[2]}

%%%%%%%%%%%%%%%%%%%%%%%% End Helper Commands %%%%%%%%%%%%%%%%%%%%%%%%%%%

%%%%%%%%%%%%%%%%%%%%%%%%% Begin CV Document %%%%%%%%%%%%%%%%%%%%%%%%%%%%

\begin{document}
\makeheading{João Carlos Arnauth Pela}

\section{Contact Information}
%
% NOTE: Mind where the & separators and \\ breaks are in the following
%       table.
%
% ALSO: \rcollength is the width of the right column of the table 
%       (adjust it to your liking; default is 1.85in).
%
\newlength{\rcollength}\setlength{\rcollength}{1.85in}%
%
\begin{tabular}[t]{@{}p{\textwidth-\rcollength}p{\rcollength}}
Trav. do Pregoeiro         & \textit{Voice:} (+351) 926357397 \\
nº24, 3º Esq.              & \textit{E-mail:} \href{mailto:joaopela@gmail.com}{joaopela@gmail.com}\\
1600-588 Lisboa            & \textit{WWW:} \href{http://www.nfist.pt/~pela}{www.nfist.pt/\~{}pela}\\
Portugal                   & \\

\end{tabular}
 
\section{Citizenship}
%
Portuguese

\section{Research Interests}
%
Particle Physics, High Energy Physics, Nuclear Fusion, Detector Physics, Instrumentation

\section{Education}
%
\href{http://www.ist.utl.pt/}{\textbf{Instituto Superior Técnico}}, 
Lisboa, Portugal
\begin{outerlist}

\item[] M.S., 
        \href{http://www.fisica.ist.utl.pt/}
             {Engineering Physics and Technology},Oct 2010
        \begin{innerlist}
        \item With final grade of 18 (in 20)
        \item Thesis Topic: Multi-Lepton Final States in the Search for New Physics at the Large Hadron Collider
        \item Supervisor: 
              \href{http://www.lip.pt/~varela/}
                   {Professor João Varela}
        \item Co-Supervisor: 
              \href{http://www.lip.pt/~michgall/}
                   {Michele Gallinaro}
        \item Area of Study: Experimental Particle Physics
        \end{innerlist}

\item[] B.S., 
        \href{http://www.fisica.ist.utl.pt/}
             {Engineering Physics and Technology}, Feb 2009
        \begin{innerlist}
        \item With final grade of 14 (in 20)
        \item Engineering specialization (emphasis on Instrumentation)
        \end{innerlist}

\end{outerlist}

\section{Complementary Education}

\href{http://www.ist.utl.pt/}{\textbf{Instituto Superior Técnico}}, 
Lisboa, Portugal
\begin{outerlist}
\item[] Level 1 Mechanical Workshop Course, Jun 2010
        \begin{innerlist}
        \item Basic use and and construction techniques with Mechanical Workshop tools and machines
        \item Allows for the use of the Instituto Superior Técnico Mechanical Workshop
        \end{innerlist}
\end{outerlist}

\section{Academic Experience}
\href{http://www.ist.utl.pt/}{\textbf{Instituto Superior Técnico}}, 
Lisboa, Portugal
\begin{outerlist}
\item[] \textit{Graduate Student}%
        \hfill \textbf{Feb 2009 to present}

\item[] \textit{Undergraduate Student}%
        \hfill \textbf{September 2002 to February 2009}

\item[] \href{http://nfist.pt/}
           {NFIST (IST Physics Students Association) Member}
\hfill \textbf{September 2002 to present}

\begin{innerlist}
  \item Development and implementation of School Registration Web Application for \href{http://sf11.nfist.pt/}{\textit{Physics Week 11}} and \href{http://sf12.nfist.pt/}{\textit{Physics Week 12}} at IST
  \item Development and implementation of Members Management Application for NFIST
  \item Development and implementation of an integrated system of mailing lists, library and forum
  \item Responsible for many activities over the years
  \item Responsible and developer of many activities Web Site (Social Activities, Outreach).
\end{innerlist}

\item[] \href{http://enef2008.nfist.pt/}{National Physics Students Meeting 2008}
\begin{innerlist}
  \item Member of the organization committee
  \item Development and implementation of all the website's applications, 
including Student Registration, Forum, News Portal, etc
\end{innerlist}

\end{outerlist}

\section{Publications}
%
\begin{innerlist}
\item CMS Exotica PAG: E. Barberis, O. Boeriu, J. Brooke, G. Bruno, J. Chen, J.P. Chou, K. F. Chen, A. De Roeck, 
S. Eno, M. Gallinaro, S. Harper, C. Hill, G. Landsberg, M. Mozer, J. Pela, S. Rappoccio, 
A. Rizzi, K. Rossato, P. Rumerio, A. Safonov, C. Shepherd-Themistocleous, 
\emph{``Data Flow for the CMS Exotica Group for Early Running``} CMS AN - 2009/XX
(Under Preparation)
\item P. Ribeiro, M. Gallinaro, M. Kazana, J. Pela, J. Varela, \emph{“Discovery potential for Universal Extra
Dimensions in the four leptons final state in pp collisions at $\sqrt{s}$ = 14 TeV”} CMS AN - 2008/035 
(\underline{Pre-approved} waiting Review Committee Final Approval)
\end{innerlist}

\section{Conference Presentations}
%
\href{http://pasc.ist.utl.pt/winterschool2008/}{PASC Winter School 2008} - Discovery potential for Extra Dimensions in the four lepton final state at the LHC

\section{Scientific Professional Experience}
%
\href{http://www.lip.pt/}{\textbf{Laboratório de Instrumentação e Física Experimental de Partículas}},
Lisboa, Portugal
\begin{outerlist}

\item[] \textit{Junior Researcher for the CMS Experiment at the LHC}%
        \hfill \textbf{2007 to present}
\begin{innerlist}
\item Coordinator and developer of the EXOTrilepton Exercise for CMS Exotica group as a part of 2009 October Exercise with the objective of testing LHC Data Flow Model
\hfill \textbf{Oct 2009}
\item Worked on 16 shifts as CMS-ECAL Barrel, Endcaps and Safety Shifter, during the CRAFT 2009 (cosmics) program. 
\hfill \textbf{Aug 2009}
\item Design and implementation of method for studying potential for Universal Extra Dimensions with a four lepton final state
\hfill \textbf{Dec 2008 to present}
\item Design and implementation of method for studying $t\bar{t}$ in semileptonic and fully leptonic with hadronic $\tau$ channels
\hfill \textbf{Set 2007 to Dec 2008}
\end{innerlist}

\end{outerlist}

\blankline

\href{http://www.cfn.ist.utl.pt/}{\textbf{Centro de Fusão Nuclear}}, 
Lisboa, Portugal
\begin{outerlist}

\item[] \textit{Experimental Physist}%
        \hfill \textbf{2003}
\begin{innerlist}
\item Helped to design a test PCB Board to implement the new system of slow control of
the ISTOK Tokamak Fusion Reactor based in CAN technology  
\end{innerlist}

\end{outerlist}

\section{Other Professional Experience}

\begin{innerlist}
\item Web Designer, designed commercial web sites for 2 small companies
\item Judo Teacher, during a period of 2 years in two separated schools
\item Sysadmin, for a local sports club during 2 years
\item Counselor in Summer Camp, during a single summer (also supported Judo classes in that camp)
\end{innerlist}

\section{Language Skills}
%
\begin{outerlist}
\item[] Portuguese
        \begin{innerlist}
        \item Native Speaker
        \end{innerlist}

\item[] English
        \begin{innerlist}
        \item \href{http://www.ielts.org}{IELTS Certification} at 30/Jan/2010 with scores: Listening 7.0, Reading 7.5,
              Writing 6.5, Speaking 7.0, Overall Band Score 7.0 (Good User)
        \item 1st language of choice during undergraduate studies (7 years of classes)
        \item Language used in work place for since 2007 
        \end{innerlist}

\item[] French
        \begin{innerlist}
        \item Generic comprehension of spoken and written language
        \item 2nd language of choice during undergraduate studies (3 years of classes) 
        \end{innerlist}

\item[] Spanish
        \begin{innerlist}
        \item Generic comprehension of spoken and written language
        \end{innerlist}
\end{outerlist}

\section{Technical Skills} 

Extensive hardware and software experience in networking and
        information technology

\blankline

Programming: C, C++, JAVA, Fortran 95, Python, Mathematica, Assembler

Database: MySQL, SQL

Web Oriented: HTML, PHP, CSS, Javascript, XML

System Administration: Shell Scripting (Bash, SH Shell), CVS

\blankline

Applications: \LaTeX{}, Microsoft Office, Open Office,
        and other common productivity packages for Windows and
        Linux platforms

\blankline

Operating Systems: Linux (administration knowledge), Microsoft Windows XP/2000/Vista/7

\blankline

Driver's Licence for motorcycle (all A class) and cars (B class)

\blankline

\end{document}

\section{Sports Experience}
%
\begin{outerlist}
  \item[] Judo
  \begin{innerlist}
    \item Graduation: $2^{nd}$ Dan - $2^{nd}$ Degree black belt \hfill \textbf{2007}
    \item Teaching: Schools and Summer Camps
    \item Refereeing: Regional Referee \hfill \textbf{2008}
    \iten Competition: $3^{rd}$ National Kata Championship 2007, $2^{nd}$ National Univercitary Championship 2005-06, $3^{rd}$ Place, National Univercitary Championship 2004-05
  \end{innerlist}
  \item[] Other Martial Arts (sporadically)
  \begin{innerlist}
    \item Martial Arts: Aikido, Iaido, Ju-Jitsu
    \item Table Tennis
    \item Swimming
  \end{innerlist}
\end{outerlist}

\blankline



%%%%%%%%%%%%%%%%%%%%%%%%%% End CV Document %%%%%%%%%%%%%%%%%%%%%%%%%%%%%
